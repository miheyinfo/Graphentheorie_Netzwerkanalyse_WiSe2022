\chapter{Results}
\label{cha:results}
\subsubsection*{Mikhail Poberezhnyi(s0558681), Sergey Wolf(s0553749)}

\subsection{Conclusion}
\label{cha:conclusion}
The goal of this paper, was to verify two hypotheses using data gathered from the Alpha Vantage financial API. This data was subsequently prepared and used to build a suitable model to use with the graph based database Neo4J. \\
\\
Our analysis revealed that the first hypothesis, which stated that media covers stocks based on their industry sector was supported by the data, as there was a clear connection between companies mentioned in news articles and their respective industry sectors. In other words, the media coverage was consistent with the industry of the covered stocks.\\
\\
However, the second hypothesis was not supported by the data. Upon analyzing the stock prices in relation to time frame before and after publication, we did not observe a clear relationship between news sentiment and stock price. The stock price seemed to fluctuate randomly, with no discernible trend or correlation to news sentiment. In conclusion, while the media appears to cover stocks based on their industry sector, there is no clear relationship between news sentiment and stock price performance.



\subsection{Future Work}
\label{cha:future-work}

The scope of this research could be expanded by incorporating additional data for the initial analysis. The second hypothesis was not supported by our findings, as the stock price appeared to fluctuate unpredictably and did not demonstrate a clear relationship with news sentiment. It is possible that a more comprehensive dataset spanning a longer timeframe may yield different results





